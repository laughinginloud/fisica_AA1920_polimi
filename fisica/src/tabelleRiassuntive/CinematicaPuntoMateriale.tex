\documentclass[../../dimostrazioni]{subfiles}

\begin{document}

    \chapter{Cinematica del punto materiale}

        \section*{Moto rettilineo Uniforme}
        
            \[    s(t) = s_0 + v_0 t   \]
            \[    v(t) = v_0     \]
            \[    a(t) = 0     \]   
        
        \section*{Moto rettilineo Uniformemente Accelerato}
        
            \[    s(t) = s_0 + v_0 t + 1/2 a t^2   \]
            \[    v(t) = v_0 + a t     \]
            \[    a(t) = a     \]
            \[    v^2 = v_0 + 2a(s - s_0)   \]
        
        \section*{Moto armonico}
        
            \[    \frac{d^2x}{dt^2} + \frac{k}{m}x = 0   \]
            \[    x(t) = A \cos(\omega t + \phi)     \]
            \[    v(t) = -A \omega \sin(\omega t + \phi)     \]
            \[    a(t) = -A \omega^2 \cos(\omega t + \phi)   \]
        
        \section*{Moto Circolare Uniforme}
        
            \[    v = \omega r   \] %Vel angolare
            \[    v = \frac{s}{t}    \] %Vel tangenziale
            \[    \omega = \frac{2\pi}{T}      \]
            \[    a_c = \omega^2 r   \]
            \[    a_c = \frac{v^2}{r}   \]
        
        \section*{Altri moti}
        
            \[    s(t) = s_0 + \int_{t_0}^{t} v(t) dt  \]
            \[    v(t) = v_0 + \int_{t_0}^{t} a(t) dt     \]
            \[    a(t) = a (t)     \]
        

        
\end{document}