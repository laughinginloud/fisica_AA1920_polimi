% !TEX root = ../../fisica.tex

\documentclass[../../fisica]{subfiles}

\begin{document}

    \chapter{Cinematica del punto materiale}

        \renewcommand{\arraystretch}{1.7}

        \begin{tabular}{ |l|>{$\displaystyle}l<{$}|}
            \hline
            \multicolumn{2}{|c|}{Cinematica} \\
            \hline\hline
            \multirow{3}{*}{Moto rettilineo uniforme}
                & s(t) = s_0 + v_0 t \\
                \cline{2-2}
                & v(t) = v_0 \\
                \cline{2-2}
                & a(t) = 0 \\
            \hline
            \multirow{4}{*}{Moto rettilineo uniformemente accelerato}
                & s(t) = s_0 + v_0 t + 1/2 a t^2 \\
                \cline{2-2}
                & v(t) = v_0 + a t \\
                \cline{2-2}
                & a(t) = a \\
                \cline{2-2}
                & v^2 = v_0 + 2a(s - s_0) \\
            \hline
            \multirow{4}{*}{Moto armonico}
                & \dv[2]{x}{t} + \frac{k}{m}x = 0 \\
                \cline{2-2}
                & x(t) = A \cos\left(\omega t + \phi\right) \\
                \cline{2-2}
                & v(t) = -A \omega \sin\left(\omega t + \phi\right) \\
                \cline{2-2}
                & a(t) = -A \omega^2 \cos\left(\omega t + \phi\right) \\
            \hline
            \multirow{5}{*}{Moto circolare uniforme}
                & v = \omega r \\
                \cline{2-2}
                & v = \frac{s}{t} \\
                \cline{2-2}
                & \omega = \frac{2\pi}{T} \\
                \cline{2-2}
                & a_c = \omega^2 r \\
                \cline{2-2}
                & a_c = \frac{v^2}{r} \\
            \hline
            \multirow{3}{*}{Altri moti}
                & s(t) = s_0 + \int_{t_0}^{t} v(t) dt \\
                \cline{2-2}
                & v(t) = v_0 + \int_{t_0}^{t} a(t) dt \\
                \cline{2-2}
                & a(t) = a (t) \\
            \hline
        \end{tabular}

\end{document}