% !TEX root = ../../fisica.tex

\documentclass[../../fisica]{subfiles}

\begin{document}

    \chapter{Dinamica dei sistemi di particelle}

        \renewcommand{\arraystretch}{1.7}

        \begin{tabular}{ |l|>{$\displaystyle}l<{$}|}
            \hline
            \multicolumn{2}{|c|}{Impulso} \\
            \hline\hline
            \multirow{1}{*}{Impulso con forza costante}
                & \vec{I} = \vec{F}_{m} \Delta t \\
            \hline
            \multirow{1}{*}{Impulso con forza variabile}
                & \vec{I} = \int_{t_i}^{t_f} \vec{F} dt \\
            \hline
            \multirow{1}{*}{Teorema dell'impulso}
                & \vec{I} = \delta \vec{p} \\
            \hline
            \multicolumn{2}{|c|}{Quantità di moto} \\
            \hline\hline
            \multirow{1}{*}{Principio conservazione della Quantità di moto}
                & \vec{F}_{RIS} = 0 \Rightarrow \vec{p} = costante \\ 
            \hline
            \multicolumn{2}{|c|}{Centro di massa} \\
            \hline\hline
            \multirow{3}{*}{Coordinate in 3 dimensioni}
                &  x_{CM} = \frac{\sum_{i=1}^n m_i x_i}{\sum_{i=1}^n m_i} \\
                \cline{2-2}
                &  y_{CM} = \frac{\sum_{i=1}^n m_i y_i}{\sum_{i=1}^n m_i} \\
                \cline{2-2}
                &  z_{CM} = \frac{\sum_{i=1}^n m_i z_i}{\sum_{i=1}^n m_i} \\
            \hline
            \multirow{1}{*}{Velocità}
                &  \vec{v}_{CM} = \frac{\vec{p}_{TOT}}{M} \\
            \hline
            \multirow{1}{*}{Accelerazione}
                &  \vec{a}_{CM} = \frac{\vec{F}_{TOT}}{M} \\
            \hline

        \end{tabular}

\end{document}