% !TEX root = ../../fisica.tex

\documentclass[../../fisica]{subfiles}

\begin{document}

    \chapter{Statica e dinamica del corpo rigido}

        \renewcommand{\arraystretch}{1.7}

        \begin{tabular}{ |l|>{$\displaystyle}l<{$}|}
            \hline
            \multicolumn{2}{|c|}{Densità} \\
            \hline\hline
            \multirow{1}{*}{Densità di un corpo rigido omogeneo}
                & p = \frac{M}{V} \\
            \hline
            \multirow{1}{*}{Densità superficiale}
                & p_s = \frac{M}{S} \\
            \hline
            \multirow{1}{*}{Densità lineare}
                & p_l = \frac{M}{L} \\
            \hline
            \multicolumn{2}{|c|}{Centro di massa e Momento} \\
            \hline\hline
            \multirow{1}{*}{Centro di massa}
                & \vec{r}_{CM} = \frac{1}{M} \int_{V} \vec{r}\rho dv \\ 
            \hline
            \multirow{2}{*}{Momento di una forza (momento torcente)}
                &  \vec{M} = \vec{r} \times \vec{F} \\
                \cline{2-2}
                &  M = r F \sin(\theta) \\
            \hline
            \multirow{1}{*}{Momento della forza peso}
                &  \vec{M} = \vec{r}_{CM} \times \vec{F}_{g} \\
            \hline
            \multicolumn{2}{|c|}{Momento di Inerzia} \\
            \hline\hline
            \multirow{1}{*}{Formula} % Questo è più che altro un placeholder perché non sapevo cosa mettere
                & I = \int_{V} r^2 dm \\
            \hline
            \multirow{1}{*}{Teorema di Huygens-Steiner}
                & I = I_{CM} + M d^2 \\
            \hline
            \multicolumn{2}{|c|}{Energia Cinetica} \\
            \hline\hline
            \multirow{1}{*}{Moto rotazionale puro}
                & K = \frac{1}{2} I \omega^2 \\
            \hline
            \multirow{1}{*}{Moto traslatiorio puro}
                & K = \frac{1}{2} m v^2 \\
            \hline
            \multirow{1}{*}{Moto rototraslatiorio}
                & K = \frac{1}{2} m v^2 + \frac{1}{2} I \omega^2 \\
            \hline
        \end{tabular}

\end{document}