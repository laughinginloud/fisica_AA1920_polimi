% !TEX root = ../../fisica.tex

\documentclass[../../fisica]{subfiles}

\begin{document}

    \chapter{Lavoro ed energia}

    % Ho lasciato la versione originale per poter recuperare più facilmente i commenti

%        \section*{Lavoro}
%        
%            \[ L = \vec{F} \vec{s}  \] % Lavoro con forza costante in una dimensione 
%            \[ L = \int_A^B\vec{F} \, d\vec{r}  \] % Lavoro in generale
%            \[ L_{p} = -mg(y_f - y_i)  \] % Lavoro forza peso
%            \[ L_{el} = - \frac{1}{2} k (x_f^2 - x_i^2)  \] % Lavoro forza elastica (x_i = s_i - s_0 ; x_f = s_f - s_0) s_0 posizione di equilibrio; s_i posizione iniziale; s_f posizione finale
%           
%        \section*{Energia}
%
%            \[ K = \frac{1}{2} m v^2  \] % Enegia cinetica
%            \[ L = K_f - K_i  = \frac{1}{2} m v_f^2 - \frac{1}{2} m v_i^2 \] % Teorema Energia Cinetica
%            \[ \Delta U = - L \] % Energia Potenziale (solo forze conservative)
%            \[ U_g = m g h \] % Energia potenziale gravitazionale bisogna decidere se tenere il pedice p (forza peso) o g e in caso mettere quello della forza peso uguale
%            \[ U_{el} = \frac{1}{2} k x^2 \] % Energia potenziale elastica
%            \[ E = K + U \] % Enegia meccanica. Bisogna decidere se tenere solo E oppure E_M
%            \[ E_i = E_f \] % Principio Conservazione Energia Meccanica assenza di forze non conservative
%
%        \section*{Potenza e quantità di moto}
%
%            \[ P = \frac{L}{t} \] % Potenza
%            \[ P = \vec{F} \vec{v} \] % Potenza con forza costante
%            \[ \vec{p} = m\vec{v} \] % Quantità di moto
%            \[ \vec{F}_{RIS} = \dv{\vec{p}}{t}\] % II PDN

        \renewcommand{\arraystretch}{1.5}

        \centering
        \begin{tabular}{ ||l|>{$}l<{$}|| }
            \hline
            \multicolumn{2}{|c|}{Lavoro ed energia} \\
            \hline\hline
            \multirow{4}{*}{Lavoro}
                & L = \vec{F} \vec{s} \\
                \cline{2-2}
                & L = \int_A^B\vec{F} \, \dd{\vec{r}} \\
                \cline{2-2}
                & L_{p} = -mg(y_f - y_i) \\
                \cline{2-2}
                & L_{el} = - \frac{1}{2} k (x_f^2 - x_i^2) \\
            \hline
            \multirow{7}{*}{Energia}
                & K = \frac{1}{2} m v^2 \\
                \cline{2-2}
                & L = K_f - K_i  = \frac{1}{2} m v_f^2 - \frac{1}{2} m v_i^2 \\
                \cline{2-2}
                & \Delta U = - L \\
                \cline{2-2}
                & U_g = m g h \\
                \cline{2-2}
                & U_{el} = \frac{1}{2} k x^2 \\
                \cline{2-2}
                & E = K + U \\
                \cline{2-2}
                & E_i = E_f \\
            \hline
            \multirow{4}{*}{Potenza e quantità di moto}
                & P = \frac{L}{t} \\
                \cline{2-2}
                & P = \vec{F} \vec{v} \\
                \cline{2-2}
                & \vec{p} = m\vec{v} \\
                \cline{2-2}
                & \vec{F}_{RIS} = \dv{\vec{p}}{t} \\
            \hline
        \end{tabular}

\end{document}