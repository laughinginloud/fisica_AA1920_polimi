\documentclass[../../dimostrazioni]{subfiles}

\begin{document}

    \chapter{Lavoro ed Energia}

        \section*{Lavoro}
        
            \[ L = \vec{F} \vec{s}  \] % Lavoro con forza costante in una dimensione 
            \[ L = \int_{A}^{B}\vec{F} \, d\vec{r}  \] % Lavoro in generale
            \[ L_{p} = -mg(y_f - y_i)  \] % Lavoro forza peso
            \[ L_{el} = - \frac{1}{2} k (x_f^2 - x_i^2)  \] % Lavoro forza elastica (x_i = s_i - s_0 ; x_f = s_f - s_0) s_0 posizione di equilibrio; s_i posizione iniziale; s_f posizione finale
           
        \section*{Energia}

            \[ K = \frac{1}{2} m v^2  \] % Enegia cinetica
            \[ L = K_f - K_i  = \frac{1}{2} m v_f^2 - \frac{1}{2} m v_i^2 \] % Teorema Energia Cinetica
            \[ \Delta U = - L \] % Energia Potenziale (solo forze conservative)
            \[ U_g = m g h \] % Energia potenziale gravitazionale bisogna decidere se tenere il pedice p (forza peso) o g e in caso mettere quello della forza peso uguale
            \[ U_{el} = \frac{1}{2} k x^2 \] % Energia potenziale elastica
            \[ E = K + U \] % Enegia meccanica. Bisogna decidere se tenere solo E oppure E_M
            \[ E_i = E_f \] % Principio Conservazione Energia Meccanica assenza di forze non conservative

        \section*{Potenza e quantità di moto}

            \[ P = \frac{L}{t} \] % Potenza
            \[ P = \vec{F} \vec{v} \] % Potenza con forza costante
            \[ \vec{p} = m\vec{v} \] % Quantità di moto
            \[ \vec{F}_RIS = \frac{d^\vec{p}}{dt}\] % II PDN

\end{document}