% Imposto la radice del documento, utile per Visual Studio Code ed altri editor
% !TEX root = ../fisica.tex

% Costanti
\newcommand{\autori}{Virginia Longo, Giovanni Manfredi, Mattia Martelli}
\newcommand{\titolo}{Appunti di Fisica}

% Utilizzo il tipo libro impostato per evitare pagine vuote, utili solo per la stampa
\documentclass[oneside]{book}

% Imposto lo stile della numerazione della pagina
\pagestyle{plain}

% Utilizzo il pacchetto per le traduzioni ed imposto la lingua
\usepackage{polyglossia}
\setmainlanguage{italian}

% Imposto il margine delle pagine
\usepackage[margin=1in]{geometry}

% Importo i pacchetti per le immagini con descrizioni, che torneranno utili per le dimostrazioni contenenti disegni
\usepackage{graphicx,subcaption}

% Importo i pacchetti per disegnare
\usepackage{tikz,pgfplots}

% Importo i pacchetti per la matematica e fisica
\usepackage{mathtools,amssymb,physics}

% Imposto i collegamenti
\usepackage{hyperref}
\hypersetup{
    unicode=true,
    bookmarksnumbered=true,
    bookmarksopen=false,
    hidelinks,
    pdftitle={\titolo},
    pdfauthor={\autori}
}

% Importo il pacchetto per lo stile degli elenchi
\usepackage{enumitem}

% Importo i pacchetti per le tabelle
\usepackage{array,multirow,longtable}

% Importo il pacchetto per le frazioni in diagonale
\usepackage{xfrac}

% Importo il pacchetto per la personalizzazione del titolo e lo modifico
\usepackage{titlesec}
\titleformat{\chapter}[display]{\normalfont\LARGE\bfseries}{Dimostrazione numero \thechapter}{.5em}{\Huge}

% Importo il pacchetto per i sottofile
\usepackage{subfiles}

% Importo il paccheto per racchiudere il testo in riquadri
\usepackage{framed}

% Imposto le informazioni presenti sulla copertina
% Nota: date rimuove la propria spaziatura, dato che non deve essere presente
\title{\titolo}
\author{\autori}
\date{\vspace{-5ex}}